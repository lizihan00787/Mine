\documentclass[UTF8]{ctexart}
% \author{\huge{第一试}}
\title{\huge{NOI 2025 模拟赛}}
\date{\Large{测试时间:2025.3.13}}
\usepackage{listings}
\usepackage{amsmath}
\usepackage{fontspec}
\usepackage{geometry}
\usepackage{listings}
\usepackage{xcolor}
\usepackage{array}
\usepackage{booktabs}
\usepackage{graphicx}
\usepackage{enumerate}
\usepackage{enumitem}
\usepackage{multirow}
\usepackage{tabularx}
\usepackage{CJK}
\usepackage{CJKfntef}
\usepackage{lastpage}
\usepackage{color}
\usepackage{fancyhdr}
\usepackage{makecell}
\newfontfamily\Consolas{Consolas}
\newcommand\file[1]{\textbf{\textit{#1}}}
\newcommand\cfile[1]{\Consolas{#1}}
\newcommand\bfdot[1]{\textbf{\CJKunderdot{#1}}}
% \pagestyle{fancy}

% \fancyhead[L]{\footnotesize \@title}
% \fancyhead[R]{\footnotesize\MakeLowercase{\fancyplain{}{\leftmark}}}

\lstset{
	basicstyle={                                % 设置代码格式
		\fontspec{Consolas}
		\normalsize
	},
	keywordstyle={                              % 设置关键字格式
		\color[RGB]{40,40,255}
		\fontspec{Consolas Bold}
		\normalsize
	},
	stringstyle={                               % 设置字符串格式
		\color[RGB]{128,0,0}
		\fontspec{Consolas}
		\normalsize
	},
	commentstyle={                              % 设置代码注释的格式
		\color[RGB]{0,96,96}
		\fontspec{Consolas}
		\normalsize
	},
	numberstyle={                               % 设置行号格式
		\normalsize
		\color{black}
		\fontspec{Consolas}
	},
	emphstyle=\color[RGB]{112,64,160},          % 设置强调字格式
	language=c++,                               % 设置语言
	numbers=left,                               % 显示行号
	%   numbersep=5pt,                              % 设置行号与代码的距离
	frame=single,                               % 设置背景边框
	tabsize=4,                                  % 设置tab长度
	backgroundcolor=\color[RGB]{245,245,244},   % 设定背景颜色
	showstringspaces=false,                     % 不显示字符串中的空格
	showspaces=false,                           % 不显示代码中的空格
	showtabs=false,                             % 不显示代码中的tab
	breaklines=true,                            % 设置自动换行
	morekeywords={},                            % 可以手动添加关键字
	emph={scanf,printf},                        % 可以手动添加强调字
	%   xleftmargin=2em,                            % 设置左边距,默认与页芯等宽
	%   xrightmargin=2em,                           % 设置右边距,默认与页芯等宽
	%   aboveskip=1em,                              % 设置上边距
}
\geometry{left=2.5cm,right=2.5cm,top=2.54cm,bottom=2.54cm}
\lstset{
	basicstyle={                                % 设置代码格式
		\fontspec{Consolas}
		\normalsize
	}
}
\setcounter{secnumdepth}{0}
\pagestyle{fancy}
\lhead{NOI 2025 模拟赛}
% \rhead{第一试}
\rhead{\MakeLowercase{\leftmark}}
\cfoot{第 \thepage 页 \hspace{2em} 共 { \color{blue} \pageref{LastPage}} 页}
\newcommand{\newproblem}[1]{\section{#1}\markboth{#1}{}}
\begin{document}
	%	\setmainfont{Times New Roman}
	%	\maketitle
	%	\huge{CSP-J/S 2022 \textbf{模拟测试}}
	\large
	\thispagestyle{empty}
	\begin{center}
		{\huge{\textbf{NOI 2025 模拟赛}}}

		~\par
		
		{\Large{CDQZ}}
		%		
		~\par
		
		{\Large{\textbf{测试时间:2025.06.16}}}
		
		~\par
		
		~\par
		
		\begin{tabularx}{\linewidth}{|X|X|X|X|}
			\hline
			题目名称 & 卡德 & 卡特 & 秋思 \\
			\hline
			题目类型 & 传统型 & 传统型 & 传统型 \\
			\hline
			目录 & \cfile{card} & \cfile{cat} & \cfile{choose} \\
			\hline
			可执行文件名 & \cfile{card} & \cfile{cat} & \cfile{choose} \\
			\hline
			输入文件名 & \cfile{card.in} & \cfile{cat.in} & \cfile{choose.in}\\
			\hline
			输出文件名 & \cfile{card.out} & \cfile{cat.out} & \cfile{choose.out}\\
			\hline
			提交文件名 & \cfile{card.cpp} & \cfile{cat.cpp} & \cfile{choose.cpp}\\
			\hline
			时间限制 & 1.0 秒 & 1.0 秒 & 2.0 秒 \\
			\hline
			内存限制 & 1024 MiB & 512 MiB & 512 MiB \\ 
			\hline
			子任务数目 & 20 & 10 & 7\\
			\hline
			测试点是否等分 & 是 & 是 & 否\\
			\hline
			编译选项 & \multicolumn{3}{c|}{\Consolas{-O2 -std=c++14}} \\ 
			\hline
		\end{tabularx}
	\end{center}
	
	%	提交源程序文件名
	
	\subsection[注意事项(请仔细阅读)]{【注意事项(请仔细阅读)】}
	
	1. 选手提交的源程序请\bfdot{直接放在个人目录下},无需建立子文件夹;
	
	2. 文件名(包括程序名和输入输出文件名)必须使用英文小写。
	
	3. C++ 中函数 main() 的返回值类型必须是 int,值必须为 0。
	
	4. \bfdot{对于因未遵守以上规则对成绩造成的影响,相关申诉不予受理}。
	
	5. 若无特殊说明,结果比较方式为\bfdot{忽略行末空格、文末回车后的全文比较}。
	
	6. 程序可使用的栈空间大小与该题内存空间限制一致。
	
	7. 在终端中执行命令 \texttt{ulimit -s unlimited} 可将当前终端下的栈空间限制放大,但你使用的栈空间大小不应超过题目限制。
	
	8. 若无特殊说明,每道题的\bfdot{代码大小限制为 100KB}。
	
	9. 若无特殊说明,输入与输出中同一行的相邻整数、字符串等均使用一个空格分隔。
	
	10. 输入文件中可能存在行末空格,请选手使用更完善的读入方式(例如 scanf 函数)避免出错。
	
	11. 直接复制 PDF 题面中的多行样例,数据将带有行号,建议选手直接使用对应目录下的样例文件进行测试。
	
	12. 使用 std::deque 等 STL 容器时,请注意其内存空间消耗。
	
	13. 请务必使用题面中规定的的编译参数,保证你的程序在本机能够通过编译。此外\bfdot{不允许在程序中手动开启其他编译选项},一经发现,本题成绩以 0 分处理。
	
	14. 统一评测时采用的机器配置为:12th Gen Intel(R) Core(TM) i7-12700 2.10 GHz,内存 16GB。上述时限以此配置为准。
	
	15. 评测在 Windows 11 下进行,使用 LemonLime 进行评测。如果对此条以及 14 条中的机器配置有疑问,请及时询问。

	16. \bfdot{题目按照字典序排序,不保证题目按照难度排序。}
	
	\newpage
	
	\newproblem{卡德(card)}

	\subsection[题目背景]{【题目背景】}

    Hanghang 在一个巨大的数轴上。
	
	\subsection[题目描述]{【题目描述】}

	给定一个长度为 $n$ 的正整数序列 $a$,设 $S = n+\sum\limits_{i=1}^n a_i$。小 H 有 $S$ 张卡片,每张卡片上都写着一个数,其中 $n$ 张上面分别写着 $a_1,a_2,\ldots,a_n$,其余 $\sum\limits_{i=1}^n a_i$ 张卡片上都写着 $-1$。
	
	小 H 现在站在数轴的坐标 $0$ 处,他将执行以下操作 $S$ 次:

	\begin{itemize}
		\item 假设小 H 现在站在坐标 $x$ 处,他会选择一张卡片并将其丢弃,设这张卡片上的数为 $x$,则小 H 会跳到坐标 $x+v$ 处。如果他刚好跳到了坐标 $0$ 处,则会获得一枚硬币。
	\end{itemize}

	现在对于所有 $k\in [1,n]$,你需要求出有多少种选牌的顺序,使得小 H 最终恰好会获得 $k$ 枚硬币,答案对 $998244353$ 取模。

	注意,对于两张写的数相同的牌,先选其中一张和先选另一张算同一种方案。
	
	\subsection[输入格式]{【输入格式】}
	
	从 \file{card.in} 中读入数据。

	第一行一个正整数 $n$。

	第二行 $n$ 个正整数,表示 $a_1,a_2,\ldots,a_n$。
	
	\subsection[输出格式]{【输出格式】}
	
	输出到文件 \file{card.out} 中。

	输出 $n$ 行,每行一个数。第 $i$ 行的数表示当 $k=i$ 时的方案数对 $998244353$ 取模的结果。
	
	\subsection[样例 1 输入]{【样例 1 输入】}
	
	\begin{lstlisting}
2
1 1
	\end{lstlisting}
	
	\subsection[样例 1 输出]{【样例 1 输出】}
	
	\begin{lstlisting}
2
4
	\end{lstlisting}
	
	\subsection[样例 1 解释]{【样例 1 解释】}

	共有 $4$ 张卡牌,上面的数分别为 $1,1,-1,-1$。一共有 $6$ 种不同的方案,以下两种方案会恰好获得 $1$ 枚硬币:$1,1,-1,-1$ 和 $-1,-1,1,1$,其余 $4$ 种方案都会恰好获得 $2$ 枚硬币。

	\subsection[样例 2 输入]{【样例 2 输入】}
	
	\begin{lstlisting}
3
1 2 3
	\end{lstlisting}
	
	\subsection[样例 2 输出]{【样例 2 输出】}
	
	\begin{lstlisting}
140
220
144
	\end{lstlisting}
	
	\subsection[样例 3]{【样例 3】}
	
	见选手目录下的 \file{card/card3.in} 与 \file{card/card3.ans}。
	
	该样例满足测试点编号 $2\sim 3$ 的限制条件。
	
	\subsection[样例 4]{【样例 4】}
	
	见选手目录下的 \file{card/card4.in} 与 \file{card/card4.ans}。
	
	该样例满足测试点编号 $7\sim 8$ 的限制条件。
	
	\subsection[样例 5]{【样例 5】}
	
	见选手目录下的 \file{card/card5.in} 与 \file{card/card5.ans}。
	
	该样例满足测试点编号 $9\sim 10$ 的限制条件。
	
	\subsection[样例 6]{【样例 6】}
	
	见选手目录下的 \file{card/card6.in} 与 \file{card/card6.ans}。
	
	该样例满足测试点编号 $13\sim 14$ 的限制条件。
	
	\subsection[样例 7]{【样例 7】}
	
	见选手目录下的 \file{card/card7.in} 与 \file{card/card7.ans}。
	
	该样例满足测试点编号 $17\sim 18$ 的限制条件。
	
	\subsection[样例 8]{【样例 8】}
	
	见选手目录下的 \file{card/card8.in} 与 \file{card/card8.ans}。
	
	该样例满足测试点编号 $19\sim 20$ 的限制条件。
	
	\subsection[测试点约束]{【测试点约束】}

	对于 $100\%$ 的数据,满足 $1\le n \le 5000,1\le a_i\le 10^9$。
	
	\begin{center}	
		\begin{tabular}{c|c|c}
			\hline
			\Xhline{1.25pt}
			测试点编号      & $n\le$               & $a_i\le$   \\ \hline
			   $1$         & $5$                  & $2$            \\ \hline
			   $2\sim 3$   & $10$                 & $5$				 \\ \hline
			   $4\sim 6$   & $12$                 &  \multirow{2}*{$10^9$} \\
			   \cline{1-2}
			   $7\sim 8$   & $20$                 &  ~              \\ \hline
			   $9\sim 10$  & \multirow{2}*{$500$} &  $1$              \\
			   \cline{1-1}\cline{3-3}
			   $11\sim 12$ &  ~                   &  $2$           \\ \hline
			   $13\sim 14$ &  $50$                & $50$            \\ \hline
			   $15\sim 16$ &  $150$               & $150$            \\ \hline
			   $17\sim 18$ &  $500$               & $500$             \\ \hline
			   $19\sim 20$ &  $5000$              & $10^9$             \\ \hline
			\Xhline{1.25pt}
		\end{tabular}
	\end{center}

	\newpage
	
	\newproblem{卡特(cat)}
	
	\subsection[题目描述]{【题目描述】}

    有一天,无聊的猫猫们想出了一个新游戏。由一只睿智的猫给出了一个长度为 $n$ 的排列,它们想知道这个排列是不是一个好排列。

	一个排列是好的,当且仅当对于所有正整数 $m$ 都满足,排列中所有长度为 $2m+1$ 的子串的\bfdot{中位数}都不在这个子串的第 $m+1$ 个位置。

	但是,这个给出排列太长了,小猫只记住了一些位上的数。所以它会给你一个一些数被替换成 $−1$ 的排列,你需要将 $−1$ 填入所有可能的值后,统计好的排列数量。

	答案对 $10^9+7$ 取模。

	\subsection[输入格式]{【输入格式】}
	
	从文件 \file{cat.in} 中读入数据。	
	
	第一行一个正整数  $t$,表示有 $t$ 组数据。

	接下来对于每组数据,第一行一个正整数 $n$。

	第二行 $n$ 个整数,表示一个长为 $n$ 的排列。
	
	\subsection[输出格式]{【输出格式】}
	
	输出到文件 \file{cat.out} 中。
	
	对于每组数据,输出一行一个整数表示可能的好的排列数量,答案对 $10^9+7$ 取模的结果。
	
	\subsection[样例 1 输入]{【样例 1 输入】}
	
	\begin{lstlisting}
5
2
-1 -1
3
-1 -1 -1
4
1 2 3 4
6
-1 -1 3 4 -1 -1
8
-1 -1 -1 -1 -1 -1 -1 -1
	\end{lstlisting}
	
	\subsection[样例 1 输出]{【样例 1 输出】}
	
	\begin{lstlisting}
2
4
0
1
316
	\end{lstlisting}

	\subsection[样例 2 输入]{【样例 2 输入】}
	
	\begin{lstlisting}
7
6
-1 -1 2 -1 -1 -1
3
-1 -1 -1
4
3 4 2 1
8
-1 -1 -1 -1 -1 -1 -1 -1
5
-1 -1 -1 -1 -1
7
-1 -1 -1 -1 -1 -1 -1
2
-1 -1
	\end{lstlisting}
	
	\subsection[样例 2 输出]{【样例 2 输出】}
	
	\begin{lstlisting}
13
4
0
316
24
136
2
	\end{lstlisting}

	\subsection[样例 2]{【样例 2】}
	
	见选手目录下的 \file{cat/cat2.in} 与 \file{cat/cat2.ans}。
	
	该样例满足子任务 $1$ 的限制条件。
	
	\subsection[样例 3]{【样例 3】}
	
	见选手目录下的 \file{cat/cat3.in} 与 \file{cat/cat3.ans}。
	
	该样例满足子任务 $2$ 的限制条件。
	
	\subsection[样例 4]{【样例 4】}
	
	见选手目录下的 \file{cat/cat4.in} 与 \file{cat/cat4.ans}。
	
	该样例满足子任务 $3$ 的限制条件。
	
	\subsection[测试点约束]{【测试点约束】}
	
	对于 $100\%$ 的数据,满足 $1\le t\le 10^4,2\le n\le 10^6,\sum n\le 10^6$,$a_i\in \{-1\}\cup [1,n]$。保证 $\forall 1\le i < j\le n,a_i\ne -1,a_j\ne -1$,满足 $a_i\ne a_j$。

	\begin{center}	
		\begin{tabular}{c|c}
			\hline
			\Xhline{1.25pt}
			测试点编号 & 特殊限制 \\ \hline
			   $1$  & $\forall 1\le i\le n,a_i=-1$ \\ \hline
               $2$  & $\sum n\le 8$ \\ \hline
               $3\sim 5$  & $\sum n^2\le 10^6$ \\ \hline
               $6\sim 10$  & $\sum n\le 10^6$ \\ \hline
			\Xhline{1.25pt}
		\end{tabular}
	\end{center}

	\newpage
	
	\newproblem{ 秋思(choose)}	
	
	\subsection[题目描述]{【题目描述】}

	有一张 $n$ 行 $m$ 列的网格图,记从上往下第 $i$ 行、从左往右第 $j$ 列的格子为 $(i,j)$,每个格子 $(i,j)$ 上有两个权值 $a_{i,j},b_{i,j}$。
	
	现在你要选出一些格子,使得不存在两个(上下左右)相邻的格子同时被选择,且选择的格子的 $(\sum a_{i,j})^2+(\sum b_{i,j})^2$ 最大。求出这个值最大是多少。
	
	\subsection[输入格式]{【输入格式】}
	
	从 \file{choose.in} 中读入数据。

	第一行输入两个整数 $n,m$。

	接下来 $n$ 行,每行输入 $m$ 个整数,第 $i$ 行第 $j$ 个整数表示 $a_{i,j}$。

	接下来 $n$ 行,每行输入 $m$ 个整数,第 $i$ 行第 $j$ 个整数表示 $b_{i,j}$。
	
	\subsection[输出格式]{【输出格式】}

	输出到文件 \file{choose.out} 中。
	
	输出一行一个整数表示答案。

	\subsection[样例 0 输入]{【样例 0 输入】}
	
	\begin{lstlisting}
2 2
1 2
3 4
1 3
1 2
	\end{lstlisting}
	
	\subsection[样例 0 输出]{【样例 0 输出】}
	
	\begin{lstlisting}
41
	\end{lstlisting}

	\subsection[样例 1]{【样例 1】}
	
	见选手目录下的 \file{choose/choose1.in} 与 \file{choose/choose1.ans}。
	
	该样例满足子任务编号 $1$ 的限制条件。

	\subsection[样例 2]{【样例 2】}
	
	见选手目录下的 \file{choose/choose2.in} 与 \file{choose/choose2.ans}。
	
	该样例满足子任务编号 $2$ 的限制条件。

	\subsection[样例 3]{【样例 3】}
	
	见选手目录下的 \file{choose/choose3.in} 与 \file{choose/choose3.ans}。
	
	该样例满足子任务编号 $3$ 的限制条件。

	\subsection[样例 4]{【样例 4】}
	
	见选手目录下的 \file{choose/choose4.in} 与 \file{choose/choose4.ans}。
	
	该样例满足子任务编号 $4$ 的限制条件。

	\subsection[样例 5]{【样例 5】}
	
	见选手目录下的 \file{choose/choose5.in} 与 \file{choose/choose5.ans}。
	
	该样例满足子任务编号 $5$ 的限制条件。

	\subsection[样例 6]{【样例 6】}
	
	见选手目录下的 \file{choose/choose6.in} 与 \file{choose/choose6.ans}。
	
	该样例满足子任务编号 $6$ 的限制条件。

	\subsection[样例 7]{【样例 7】}
	
	见选手目录下的 \file{choose/choose7.in} 与 \file{choose/choose7.ans}。
	
	该样例满足子任务编号 $7$ 的限制条件。

	\subsection[测试点约束]{【测试点约束】}

	本题使用捆绑测试。
	
    对于 $100\%$ 的数据,满足 $n,m\ge 1,1\leq n\times m\leq 3000,1\leq a_{i,j},b_{i,j}\leq 10000$。

    \begin{center}	
		\begin{tabular}{c|c|c|c}
			\hline
			\Xhline{1.25pt}
			子任务编号 & $nm\le $ & 特殊性质 & 分值 \\ \hline
			   $1$ & $16$   & 无                & 5 \\ \hline
			   $2$ & $81$   & \multirow{3}*{有} & 5 \\  
			   \cline{1-2}\cline{4-4}
			   $3$ & $196$  & ~                 & 5 \\ 
			   \cline{1-2}\cline{4-4}
			   $4$ & $3000$ & ~                 & 10 \\ \hline
			   $5$ & $36$   & \multirow{3}*{无} & 15 \\
			   \cline{1-2}\cline{4-4}
			   $6$ & $81$   & ~                 & 15 \\
			   \cline{1-2}\cline{4-4}
			   $7$ & $3000$ & ~                 & 45 \\ \hline
			\Xhline{1.25pt}
		\end{tabular}
	\end{center}

	特殊性质:$\forall 1\leq i\leq n,1\leq j\leq m,b_{i,j}=0$。

\end{document}
