\documentclass[UTF8]{beamer}
\usepackage{ctex}
\usepackage{graphicx}
\usepackage{ulem}
\usepackage{hyperref}
\usepackage{listings}
\usepackage{graphicx}

\usetheme[block=fill, sectionpage=none]{Berlin}
\usecolortheme{custom}

\title{组合数学}
\author{God\_Max\_Me}
\institute{Chengdu No.7 High School.}
\date{2025 年}
\usefonttheme[onlymath]{serif}

\geometry{paperheight=11.0cm,paperwidth=16.0cm}




\begin{document}
  \maketitle
  \section{前言}

  \begin{frame}{一些必要 trick}
    \begin{itemize}
      \item
        推式子,先提 $\sum$ 和 $\Pi$
        到最前面,然后从后往前合并,必要时考虑更改 $\sum$ 的取值
      \item
        看到次方变为斯特林数,$x^n=\sum\limits_{i=0}^{n} {n \brack i} {x \choose i}i!=\sum\limits_{i=0}^{n}\sum\limits_{i=1}^m{(-1)^{m-i}{i^n \over (m-i)!}}{x \choose i}$
      \item
        注意莫反、欧反的形式
    \end{itemize}
  \end{frame}

  \begin{frame}{推式子基本原理}
    \begin{itemize}
      \item
        先把 $\sum$ 移到最前面。
      \item
        将多个 $\sum$ 排序,范围更小的放在前面。
      \item
        将只与当前 $\sum$ 有关的式子尽量往前提。
      \item
        将能简化式子的特殊边界提出来。
      \item
        从后往前处理。
    \end{itemize}
  \end{frame}
  
  \section{组合数学}
  \subsection{组合恒等式}

  \begin{frame}
    \begin{block}{递推式}
      $$
      {n \choose m}={n-1 \choose m}+{n-1 \choose m-1}
      $$
    \end{block}
    
    \pause
    证明

    从组合意义上推导,在 $n$ 个人中选 $m$
    个相当于单独考虑最后一人,若他要选,则为 ${n-1 \choose m-1}$ 他不选则为
    ${n-1 \choose m}$。
  \end{frame}
  
  \begin{frame}
    \begin{block}{吸引/相伴等式}
      $$
      \begin{aligned}
      {{n \choose m} \over {n-1 \choose m-1}}&={{n} \over {m}} \\
      {{n \choose m} \over {n-1 \choose m}}&={{n} \over {n-m}} \\
      {{n \choose m} \over {n \choose m-1}}&={{n-m+1} \over {m}} \\
      \end{aligned}
      $$

      \pause
      另外的形式:

      $$
      \begin{aligned}
      k {n \choose k}&=n {n-1 \choose k-1} \\
      (n-k){n \choose k}&=n{n-1 \choose k}
      \end{aligned}
      $$
    \end{block}
  \end{frame}

    \begin{frame}
      \begin{block}{上指标反转}
        $$
        {n \choose m}=(-1)^m{m-n-1 \choose m}
        $$
      \end{block}
      \pause
      证明

      $$
      \begin{aligned}
      {n \choose m}={n^{\underline{m}} \over m!}&={n\times(n-1)\times (n-2)\times...\times (n-m+1) \over m!} \\
      &={(-1)^m\times (-n)\times(1-n)\times...\times(m-n-1) \over m!} \\
      &={(-1)^m\times (m-n-1)^{\underline{m}} \over m!} \\
      &=(-1)^m{m-n-1 \choose m}
      \end{aligned}
      $$
    \end{frame}

    \begin{frame}
      \begin{block}{三项式系数恒等式}
        $$
        {n \choose m}{m \choose k}={n \choose k}{n-k \choose m-k}
        $$

        等式两边拆开约分即可得证。
      \end{block}
      \pause
      \begin{block}{平行求和}
        $$
        \begin{aligned}
        {n \choose 0}&+{n+1 \choose 1}+...+{n+m \choose m} \\
        &=\sum\limits_{i=0}^{m}{n+i \choose i}={n+m+1 \choose m} \\
        m &\in \mathbb{N}
        \end{aligned}
        $$

        证明

        将${n+m+1 \choose m}$用加法公式展开即可。
      \end{block}
    \end{frame}

    \begin{frame}
      \begin{block}{上指标求和}
        $$
        \sum\limits_{i=0}^{n}{i \choose m}={n+1 \choose m+1}
        $$

        证明

        从组合意义入手,相当于我从 \(n+1\) 个数中选 \(m+1\) 个数,先假设选
        \(i\),那么 \(i\) 前面还需要选 \(m\) 个数,枚举这个 \(i\),即为答案。
        也可通过微积分求导知识进行证明,这里不再详述。
      \end{block}
    \end{frame}

    

\end{document}