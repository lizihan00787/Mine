\documentclass[UTF8]{beamer}
\usepackage{ctex}
\usepackage{graphicx}
\usepackage{ulem}
\usepackage{hyperref}
\usepackage{listings}
\usepackage{graphicx}

\usetheme[block=fill, sectionpage=none]{Berlin}
\usecolortheme{custom}

\title{组合数学1}
\subtitle{二项式系数}
\author{harryzhr}
\date{2025 年 1 月 22 日}
\usefonttheme[onlymath]{serif}

\geometry{paperheight=11.0cm,paperwidth=16.0cm}

\begin{document}
    \maketitle
    \section{组合数}
    \subsection{组合恒等式}
    \begin{frame}{定义}
        \begin{itemize}
            \item \textbf{组合数},也叫二项式系数,定义为
            $$
            C_{n}^{m}=\binom{n}{m}=\frac{n!}{m!(n-m)!}=\frac{n(n-1)\cdots(n-m+1)}{m!}
            $$
            表示从 $n$ 个不同元素中选出 $m$ 个(不计顺序)的方案数。
            \item \textbf{下降幂}:$n$ 的 $m$ 阶下降幂定义为
            $$
            n^{\underline m} = n(n-1)\cdots (n-m+1)
            $$
            这样二项式系数可以写为
            $$
            \binom{n}{m} = \frac{n^{\underline m}}{m!}
            $$
        \end{itemize}
    
    \end{frame}

    \begin{frame}{插板法}
        \begin{block}{插板法1}
            有 $n$ 个完全相同的元素,将其分为 $k$ 组,每组至少有一个元素,一共有多少种分法?
        \end{block}
        \pause
        \vspace{-0.7em}
        \begin{block}{Solution}
            $k-1$ 块板插入 $n-1$ 个空隙,答案为
            \vspace{-1em}
            $$
            \binom{n-1}{k-1}
            $$
            \vspace{-1em}
        \end{block}
        
        \pause
        \begin{block}{插板法2}
            有 $n$ 个完全相同的元素,将其分为 $k$ 组,每组可以为空,一共有多少种分法?
        \end{block}
        \pause
        \vspace{-0.7em}
        \begin{block}{Solution}
            先加 $k$ 个元素,让每组都非空,最后从每组中拿走一个元素,答案为
            \vspace{-0.5em}
            $$
            \binom{n+k-1}{k-1}=\binom{n+k-1}{n}
            $$
            \vspace{-1em}
        \end{block}
    \end{frame}
    \begin{frame}
        \begin{block}{插板法3}
            从 $1,2,\cdots,n$ 中选出 $k$ 个数,要求任何两个数都不相邻,一共有多少种选法?
        \end{block}
        \pause
        \begin{block}{Solution}
            这 $k$ 个元素把 $1\sim n$ 划分成了 $k+1$ 段,设每段的值为 $a_0,a_1,\cdots,a_{n-1},a_n$,则它们满足
            $$
            a_0+a_1+\cdots+a_n=n-k\land a_1,a_2,\cdots,a_{n-1}>0,a_0,a_{n}\ge 0
            $$
            给 $a_0,a_n$ 都加上 $1$,让 $a_0,a_1,\cdots,a_{n-1},a_n$ 都 $>0$,此时
            $$
            a_0+a_1+\cdots+a_n=n-k+2
            $$
            即将 $n-k+2$ 个数划分成至少为 $1$ 的 $k+1$ 段,方案数为
            $$
            \binom{n-k+1}{k}
            $$
        \end{block}
    \end{frame}

    \begin{frame}{定义的拓展}
        上述定义的定义域为 $n,m\in \mathbb N$,使用下面的定义,将定义域扩展到 $n\in \mathbb R,m\in \mathbb Z$:

        $$
        \binom nm = \begin{cases}
            \frac{n^{\underline m}}{m!} & m\ge 0\\
            0& m<0
        \end{cases}
        $$

        规定 $0^0=0!=x^{\underline 0}= 1$。
    \end{frame}

    \begin{frame}{组合恒等式}
        下面的恒等式如果没有特殊标注则表示对 $n\in \mathbb R,k\in \mathbb Z$ 成立
        \begin{block}{加法公式}
        $$
        \binom nk = \binom{n-1}{k}+\binom{n-1}{k-1}
        $$
        \end{block}
        \pause

        \begin{block}{对称公式}
            $$
            \binom nk = \binom n{n-k},\quad n,k\in \mathbb N
            $$

            注意对称公式对负的 $n$ 是不成立的
        \end{block}
    \end{frame}

    \begin{frame}
        \begin{block}{吸收公式}
            $$
            k\binom{n}{k}=n\binom{n-1}{k-1},\quad (n-k)\binom nk = n\binom{n-1}{k}
            $$
        \end{block}
        \begin{block}{上指标反转}
            $$
            \binom nk=(-1)^k\binom{k-n-1}k
            $$
        \end{block}
        \pause
        以上柿子均可以由定义推出,这里不再赘述。
    \end{frame}

    \begin{frame}{组合求和式}
        \begin{block}{广义二项式定理}
            对 $n\in \mathbb R$,$|\frac{x}{y}|<1$
            \vspace{-0.5em}
            $$
            (x+y)^n = \sum_{k=0}^{+\infty}\binom{n}{k}x^ky^{n-k}
            $$
            特别地,当 $n\in \mathbb N$ 时就是常用的二项式定理
            \vspace{-0.5em}
            $$
            (x+y)^n = \sum_{k=0}^{n}\binom{n}{k}x^ky^{n-k}
            $$\vspace{-1em}
        \end{block}
        证明的话对 $f(x)=(x+y)^n$ 在 $x=0$ 处做泰勒展开即可。
        \pause
        \begin{block}{组合数一行之和}\vspace{-1em}
            $$
            \sum_{k=0}^n \binom nk=2^n,\quad n\in\mathbb{N}
            $$\vspace{-1em}
        \end{block}
        二项式定理取 $x=y=1$ 即可。
    \end{frame}

    \begin{frame}

        \begin{block}{平行求和}\vspace{-2em}
            $$
            \binom{n}0+\binom{n+1}{1}+\cdots+\binom{n+m}{m} = \sum_{k=0}^m\binom{n+k}{k}=\binom{n+m+1}{m},\quad m\in \mathbb N
            $$\vspace{-1.2em}
        \end{block}
        \pause
        \begin{block}{证明}
            将 $\binom{n+m+1}{m}$ 用加法公式一步步展开即可
            $$
            \begin{aligned}
                \binom{n+m+1}m &= \binom{n+m}{m}+\binom{n+m}{m-1}\\
                &=\binom{n+m}{m}+\binom{n+m-1}{m-1}+\binom{n+m-1}{m-2}\\
                &=\cdots\\
                &=\binom{n+m}{m}+\binom{n+m-1}{m-1}+\cdots+\binom{n+1}{1}+\binom{n}0
            \end{aligned}
            $$
        \end{block}
    \end{frame}

    \begin{frame}
        \begin{block}{上指标求和}\vspace{-1em}
            $$
            \binom{0}m+\binom{1}{m}+\cdots+\binom{n}{m} = \sum_{k=0}^n\binom{k}{m}=\binom{n+1}{m+1},\quad n,m\in \mathbb N
            $$\vspace{-0.5em}
        \end{block}
        \pause
        \begin{block}{证明}
            将 $\binom{n+1}{m+1}$ 用加法公式一步步展开即可
        \end{block}
    \end{frame}
    \begin{frame}
        \begin{block}{交错和}
            \vspace{-1em}
            $$
            \sum_{k=0}^m(-1)^k\binom{n}{k} =(-1)^m\binom{n-1}{m},\quad m\in \mathbb Z
            $$\vspace{-1em}
        \end{block}
        \pause
        \begin{block}{证明}
            \vspace{-1em}
            $$
            \begin{aligned}
                \sum_{k=0}^m(-1)^k\binom{n}{k} &=\sum_{k=0}^m\binom{k-n-1}{k}&&\text{上指标反转}\\
                \pause
                &=\binom{m-n}{m}&&\text{平行求和}\\
                &=(-1)^m\binom{n-1}{m}&&\text{上指标反转}\\
            \end{aligned}
            $$
        \end{block}
    \end{frame}

    \begin{frame}
        \begin{block}{范德蒙德卷积}\vspace{-1em}
            $$
            \sum_{k}\binom{r}{k}\binom{s}{n-k}=\binom{r+s}{n},\quad n\in\mathbb Z
            $$\vspace{-1em}
        \end{block}
        \pause
        \begin{block}{组合意义}
            从 $r+s$ 个物品中一共要选 $n$ 个,总方案数是 $\binom{r+s}{n}$。枚举 $k$ 表示从前 $r$ 个物品中选了 $k$ 个,那么从后面 $s$ 个物品中就选了 $n-k$ 个,方案数是 $\binom{r}{k}\binom{s}{n-k}$。
        \end{block}
        \pause
        进一步结合其他恒等式可以得到下面这些式子
    \end{frame}

    \begin{frame}{多项式系数}
        \begin{block}{三项式版恒等式}
            $$
            \binom{r}{m}\binom{m}{k}=\binom{r}{k}\binom{r-k}{m-k},\quad m,k\in \mathbb Z
            $$
        \end{block}
        两边都是用定义展开即可得证。

        \pause

        $n$ 个两两不同的物品,分成 $m$ 组,第 $i$ 组有 $a_i$ 个物品,则总的方案数为:
        $$\frac{n!}{\prod_{i=1}^ma_i!}$$
        \pause
        这个被叫做多项式系数,记做 $\binom{n}{a_1,a_2,\cdots,a_m}=\binom{a_1+a_2+\cdots+a_m}{a_1,a_2,\cdots,a_m}$\\
        \pause
        它等于如下的二项式系数乘积:
        $$\binom{a_1+a_2+\cdots+a_m}{a_1,a_2,\cdots,a_m}=\binom{a_1+\cdots+a_m}{a_2+\cdots+a_m}\binom{a_2+\cdots+a_m}{a_3+\cdots+a_m}\cdots\binom{a_{m-1}+a_m}{a_m}\binom{a_m}0$$
    \end{frame}

    \begin{frame}
        总结一下,常用恒等式如下表所示:
    \end{frame}

    \begin{frame}
        上指标求和和平行求和都有封闭形式,但是遗憾的是,下指标求和没有封闭形式。
        $$
        \sum_{k\le m}\binom{n}{k},\quad n,m\in \mathbb N
        $$
    \end{frame}

    \begin{frame}
        \sout{开始推式子吧}
        \begin{block}{求出下式的封闭形式}\vspace{-1em}
            $$\sum_k\binom nk^2,\ n\in\mathbb{N}$$
            \vspace{-1em}
        \end{block}
        \pause
        对称公式之后范德蒙德卷积即可。
        $$\sum_k\binom nk^2=\sum_k\binom nk\binom n{n-k}=\binom{2n}n$$
    \end{frame}
    \begin{frame}
        \begin{block}{求出下式的封闭形式}
            $$\sum_kk\binom nk^2$$
        \end{block}
        \pause
        吸收律之后对称公式之后范德蒙德卷积即可。
        \begin{align*}
              & \sum_kk\binom nk^2=n\sum_k\binom nk\binom {n-1}{k-1}\\
            = & n\sum_k\binom nk\binom {n-1}{n-k}=n\binom{2n-1}{n}
        \end{align*}
    \end{frame}
    \begin{frame}
        \begin{block}{《具体数学》 的例题}
            $$\sum_{k=0}^m\frac{\binom mk}{\binom nk},\ n,m\in\mathbb{N},\ n\geq m$$
        \end{block}
        \pause
        注意到
        $$\binom nm\binom mk=\binom nk\binom{n-k}{m-k}\iff\frac{\binom mk}{\binom nk}=\frac{\binom{n-k}{m-k}}{\binom nm}$$
        提出 $\frac{1}{\binom nm}$ ,
        $$
        \begin{aligned}
            \text{原式}=\frac{1}{\binom{n}{m}}\sum_{k=0}^m\binom{n-k}{m-k}=\frac{1}{\binom{n}{m}}\sum_{k=0}^m\binom{n-m+k}{k}=\frac{\binom{n+1}{m}}{\binom nm}=\frac {n+1}{n+1-m}
        \end{aligned}
        $$
    \end{frame}

    \begin{frame}
        \begin{block}{求出下式的封闭形式}\vspace{-1em}
            $$\sum_{k=m}^n(-1)^{k}\binom nk\binom km,\ n,m\in\mathbb{N},\ n\geq m$$
        \end{block}
        \vspace{-1em}
        \pause
        $$
        \begin{aligned}
            & \sum_{k=m}^n(-1)^k\binom nk\binom km =\sum_{k=m}^n(-1)^k\binom nm\binom{n-m}{k-m}\\
            & =\binom nm\sum_{k=m}^n(-1)^k\binom{n-m}{k-m}=\binom nm(-1)^m\sum_{k=m}^n\binom {k-n-1}{k-m}\\
            & =\binom nm(-1)^m\sum_{k=0}^{n-m}\binom{k+m-n-1}{k}=\binom nm(-1)^m\binom{0}{n-m}\\
            & =(-1)^m[n=m]
        \end{aligned}
        $$
    \end{frame}

    \begin{frame}
        还有一种做法,可以考虑生成函数:
        \pause
        \begin{align*}
            & \sum_{m=0}^{n}\sum_{k=m}^n(-1)^{k}\binom nk\binom kmx^m\\
            & =\sum_{k=0}^{n}(-1)^{k}\binom nk\sum_{m=0}^k\binom kmx^m\\
            & =\sum_{k=0}^{n}(-1)^{k}\binom nk(1+x)^k\\
            & =(-1-x+1)^n=(-x)^n
        \end{align*}
        \pause
        于是上式的 $x^m$ 项系数为
        $$
        [x^m](-x)^n=[m=n](-1)^n
        $$
    \end{frame}

    \begin{frame}{例题}
        \begin{block}{《具体数学》 的例题}
            $$\sum_k\binom{n+k}{2k}\binom{2k}{k}\frac{(-1)^k}{k+1},\quad n\in \mathbb N$$
        \end{block}
        \pause
        \begin{align*}
              & \sum_k\binom{n+k}{2k}\binom{2k}{k}\frac{(-1)^k}{k+1}=\sum_k\binom{n+k}{k}\binom{n}{k}\frac{(-1)^k}{k+1}\\
            = & \sum_k\binom{n+k}{k}\binom{n+1}{k+1}\frac{(-1)^k}{n+1}=\frac{1}{n+1}\sum_k\binom{-n-1}{k}\binom{n+1}{k+1}\\
            = & \frac 1{n+1}\binom 0n=[n=0]
        \end{align*}
    \end{frame}

    \begin{frame}
        \begin{block}{《具体数学》 的例题}
            $$\sum_k\binom{n+k}{m+2k}\binom{2k}{k}\frac{(-1)^k}{k+1}$$
        \end{block}
        \pause
        利用式 (5.26) 尝试反向使用范德蒙德卷积:
        \begin{align*}
            = & \sum_k\binom{2k}{k}\frac{(-1)^k}{k+1}\sum_{0\leq j\leq n+k-1}\binom{n+k-1-j}{2k}\binom{j}{m-1}\\
            = & \sum_{j\ge 0}\binom{j}{m-1}\sum_{k\ge j-n+1}\binom{n+k-1-j}{2k}\binom{2k}{k}\frac{(-1)^k}{k+1}\\
            = & \sum_{j\ge 0<n}\binom{j}{m-1}\sum_{k\ge 0}\binom{n+k-1-j}{2k}\binom{2k}{k}\frac{(-1)^k}{k+1}\\
            = & \sum_{0\le j<n}\binom{j}{m-1}[n-1-j=0]
            = \binom{n-1}{m-1}
        \end{align*}
    \end{frame}
    
    \subsection{例题}
    \begin{frame}{例题}
        \sout{终于到OI题了}

        \begin{block}{[SDOI2016] 排列计数}
        求有多少种 $1\sim n$ 的排列 $a$,满足恰有 $m$ 个位置 $i$ 使得 $a_i = i$。答案对 $10^9 + 7$ 取模。

        $1 \leq T \leq 5 \times 10^5$,$1 \leq n \leq 10^6$,$0 \leq m \leq 10^6$。
        \end{block}
        \pause
        记 $n$ 个元素的错排为 $d_n$,那么原式就等于 $\binom nm d_{n-m}$,考虑快速计算 $d$。
    \end{frame}
    \begin{frame}
        
        一种做法是直接容斥:
        $$d_n=n!+\sum_{k=1}^n(-1)^k\binom{n}{k}(n-k)!=n!\sum_{0\leq k\leq n}(-1)^k\frac 1{k!}$$
        \pause
        另一种做法是考虑递推,考虑 $n$ 所在的环的大小:
        \begin{itemize}
            \item 大小为 $2$,那么剩下的所有是一个 $n-2$ 的错排,方案数为 $(n-1)d_{n-2}$
            \item 大小 $>2$,那么插入 $n$ 之前就已经是一个错排了,枚举插入位置,方案数为 $(n-1)d_{n-1}$
        \end{itemize}
        $$d_n=(n-1)(d_{n-1}+d_{n-2})$$
    \end{frame}

    \begin{frame}{例题}
        \begin{block}{[六省联考 2017] 组合数问题}
            求
            $$
            \sum_{i\bmod k=r}\binom{nk}{i}\bmod p
            $$
            $1\le n\le 10^9,0\le r<k\le 1000, 2\le p\le 2^{30}-1$
        \end{block}
        \pause
        $$
        \begin{aligned}
        &=\sum_{i\bmod k=r}[x^i](1+x)^{nk}\\
        &=[x^r]\Big((1+x)^{nk}\bmod (x^k-1)\Big)
        \end{aligned}
        $$

        循环卷积快速幂即可,复杂度 $O(k^2\log nk)$
    \end{frame}

%     \begin{frame}{例题}
%         \begin{block}{[HAOI2011] Problem c}

% 给 $n$ 个人安排座位,先给每个人一个 $1\thicksim n$ 的编号,第 $i$ 个人的编号为 $a_i$(不同人的编号可以相同)。接着从第一个人开始,大家依次入座,第 $i$ 个人来了以后尝试坐到 $a_i$,如果 $a_i$ 被占据了,就接着依次尝试 $a_i+1,a_i+2,\cdots,n$。如果一直尝试到第 $n$ 个都不行,该安排方案就不合法。

% 有 $m$ 个人的编号已经确定,你只能安排剩下的人的编号,求有多少种合法的安排方案。答案对 $M$ 取模。

% $1 \leq T \leq 10$,$1 \leq n \leq 300$, $0 \leq m \leq n$, $2 \leq M \leq 10^9$。
%         \end{block}
%         \pause

        
%         记 $s_i$ 表示一个方案中编号 $\ge i$ 的人的数量。那么无解当且仅当 $\exists i $ 使得 $s_i>n-i+1$。
%         DP,设 $f_{i,j}$ 表示在考虑完所有 $\ge i$ 的编号,除了那 $m$ 个确定的人以外还有 $j$ 个人的编号 $\ge i$,那么转移为
%         $$
%         f_{i,j}=\sum_{k=0}^jf_{i+1,j-k}\binom{j}{k}
%         $$
%         复杂度 $O(n^3)$,使用多项式卷积可以做到 $O(n^2\log n)$。
%     \end{frame}

    \begin{frame}{例题}
        \begin{block}{[HAOI2018] 苹果树}
            一棵二叉树,初始只有一个根节点。每次随机在可能的位置接上一个节点来产生
            一个 $n$ 个节点的二叉树,设树上节点两两距离之和的期望为 $E$,求 $n!\cdot E$ 在 $\bmod P$ 意义下的值。

            $n\le 2000, P\le 10^9+7$
        \end{block}

        \pause

        做法很多,这里随便讲一种。

        每次操作之后会减少一个剩余位置,增加两个剩余位置,这样能生成的树的总数为 $2\times 3\times \cdots \times n = n!$,于是答案为所有树的节点两两距离之和。

        
    \end{frame}

    \begin{frame}
        对于一个节点 $i$($i\ge 2$),$(i,\mathrm{father}_i)$ 这条边的贡献是 $size_i(n-size_i)$。那么枚举 $size_i$,我们要求 $i$ 的子树大小为 $size_i$ 的树的数量。
        \begin{itemize}
            \item $1\sim i$ 这些点随便连,方案数是 $i!$
            \item 从剩下的 $n-i$ 个节点里选出 $size_i-1$ 个放进 $i$ 的子树,选点方案 $\binom{n-i}{size_i-1}$,子树内方案数 $(size_i)!$
            \item 剩下的点要放到 $i$ 子树外面,方案数是 $(i-1)i(i+1)\cdots(n-size_i-1)=\frac{(n-size_i-1)!}{(i-2)!}$
        \end{itemize}
        最终答案为
        $$
        \begin{aligned}
        &\sum_{i=2}^n\sum_{size_i=1}^{n-i+1}size_i(n-size_i)\cdot i!\cdot\binom{n-i}{size_i-1}(size_i)!\frac{(n-size_i-1)!}{(i-2)!}\\
        =&\sum_{i=2}^n\sum_{size_i=1}^{n-i+1}size_i(n-size_i)\cdot\binom{n-i}{size_i-1}(size_i)!(n-size_i-1)!i(i-1)\\
        \end{aligned}
        $$
    \end{frame}

    \begin{frame}{例题}
        \begin{block}{[ZJOI2015] 地震后的幻想乡}
            给定一张 $n$ 个点 $m$ 条边的图,边的边权是 $[0,1]$ 之间均匀分布的随机实数,且相互独立。求最小生成树的最大边权的期望值。

            结果保留 $6$ 为小数。

            $n\le 10,m\le \frac{n(n-1)}{2}$

            提示(原题给的提示):对于 $n$ 个 $[0,1]$ 之间的随机变量 $x_1,\cdots,x_n$,第 $k$ 小值的期望值是 $\frac{k}{n+1}$。
        \end{block}
        \pause

        首先有一个暴力做法,枚举边权的相对大小,然后做最小生成树,kruskal算出最小生成树最大边权的排名,然后根据提示得出此时的最大边权的期望。

        这个想法启发我们钦定一个边集 $S$ 和一条边 $e$,$S$ 为前 $|S|$ 小的所有边,$e$ 为第 $|S|+1$ 小的边。如果加入 $S$ 后图未连通,加入 $e$ 后\textbf{恰好}使图联通,那么此时最小生成树的期望就是 $\frac{|S|+1}{m+1}$。$S$ 恰好是前 $|S|$ 小且 $e$ 恰好是第 $|S|+1$ 小的概率是 $\frac{1}{\binom{m}{|S|+1}\cdot (|S|+1)}$
    \end{frame}

    \begin{frame}
        于是我们统计这样的 $(S,e)$ 的数量。恰好联通这个条件并不好统计,我们转换一下,可以变成加之前不连通的边集数 $-$ 加之后不连通的边集数。

        令 $f_{S,i},g_{S,i}$ 分别表示点集为 $S$,边集大小为 $i$,且点集不连通/连通的边集数量,令 $d_S$ 表示点集 $S$ 的导出子图的边数,则
        $$
        g_{S,i}+f_{S,i} = \binom{d_S}{i}
        $$
        考虑 $f$ 的转移,对于一个 $S$,任取 $S$ 中的一个点 $k$,枚举 $k$ 所在的连通块 $T$
        $$
        f_{S,i} = \sum_{k\in T\subsetneq S}\sum_{j} g_{T,j}\binom{d_{S\setminus T}}{i-j}
        $$
    \end{frame}

    \begin{frame}
        最后考虑如何统计答案,令点集的全集为 $U$,考虑加入第 $k$ 条边时恰好连通的二元组 $(S,e)$,$|S|+1=k$,这样的二元组数量是
        $$
        (m-k+1)f_{U,k-1}-kf_{U,k}
        $$

        于是最终答案为
        $$
        \sum_{k=1}^m\frac{k}{m+1}\cdot\frac{1}{\binom{m}{k}\cdot k}((m-k+1)f_{U,k-1}-kf_{U,k})
        $$

        化简可得一个更简单的表达式为
        $$
        \frac{1}{m+1}\sum_{k=1}^m\frac{f_{U,k}}{\binom{m}{k}}
        $$

    \end{frame}

    \begin{frame}{例题}
        \begin{block}{某联考题}
            给定序列 $a$,每次询问给出 $[l,r]$ 和 $k$。
            
            回答:在序列 $a$ 中值在 $[l,r]$ 内的数所有数中随机选择 $k$ 个,最大值的期望$\bmod\  998244353$,不足 $k$ 个输出 $-1$。

            $n,\sum k\le 10^5,a_i\le 10^8$
        \end{block}

        \pause

        排序后把值域$[l,r]$换成序列上的区间$[l,r]$,答案为 
        $$
        \sum_{i=l+k-1}^r\binom{i-l}{k-1}a_i
        $$
    \end{frame}

    \begin{frame}
        
    根据范德蒙德卷积
    $$
    \begin{aligned}
    ans&=\sum_{i=l+k-1}^ra_i\sum_{j=0}^{k-1}\binom{i}{j}\binom{-l}{k-j-1}\\
    &=\sum_{j=0}^{k-1}\binom{-l}{k-j-1}\sum_{i=l+k-1}^{r}a_i\binom{i}{j}
    \end{aligned}
    $$
    前面的组合数使用上指标反转即可。

    \pause
    因为限制了 $\sum k$,考虑根号分治

    \begin{itemize}
        \item 
    $<B$的部分,对 $\sqrt{B}$个$j$,预处理出 $a_i\binom{i}{j}$ 的前缀和,询问的时候枚举 $j$ 即可,复杂度 $O(Bn+\sum k)$

    \item $\ge B$的部分,$O(n)$暴力即可
    \end{itemize}


    $B=\sqrt{n}$有最优复杂度,总复杂度 $O(n\sqrt{n})$。
    \end{frame}

    \section{Lucas 定理}
    \subsection{Lucas 定理}
    \begin{frame}{Lucas 定理}
        Lucas 定理可以用来求大组合数对小模数取模的结果
        \begin{block}{Lucas 定理}
            对于质数 $p$,
            $$
            \binom{n}{m}\bmod p = \binom{\lfloor n/p\rfloor}{\lfloor m/p\rfloor}\cdot\binom{n\bmod p}{m\bmod p}\bmod p
            $$
        \end{block}
    \end{frame}

    \begin{frame}{Lucas 定理的证明}
        \begin{block}{引理1}\vspace{-1em}
            $$
            \binom{p}{n}\bmod p = [n=0\lor  n=p]
            $$\vspace{-1em}
        \end{block}
        \textbf{证明.} $\binom{p}{n}=\frac{p!}{n!(p-n)!}$,分子中 $p$ 的次数为 $1$,若 $\binom{p}{n}\not\equiv 0 \pmod p$,则 $n=p$ 或 $p-n=p$,此时 $n=0$ 或 $p$,$\binom{p}{n}\bmod p = 1$。

        \pause

        \begin{block}{引理2}
            \vspace{-1em}
            $$
            (a+b)^p\equiv a^p+b^p \pmod p
            $$
            \vspace{-1em}
        \end{block}
        \textbf{证明.}
        \vspace{-1em}
        $$
        \begin{aligned}
        (a+b)^p&\equiv \sum_{n=0}^p\binom{p}{n}a^nb^{p-n} \pmod p\\
        &\equiv a^p+b^p \pmod p
        \end{aligned}
        $$
    \end{frame}

    \begin{frame}{Lucas 定理的证明}
        $\binom{n}{m}\bmod p$ 为 $(1+x)^n\bmod p $ 的 $x^m$ 项系数,
        $$
        \begin{aligned}
        (1+x)^n\bmod p &= (1+x)^{p\lfloor n/p\rfloor }(1+x)^{n\bmod p}\bmod p\\
        &= (1+x^p)^{\lfloor n/p\rfloor }(1+x)^{n\bmod p}\bmod p
        \end{aligned}
        $$
        $(1+x^p)^{\lfloor n/p\rfloor }$ 中的项的次数都是 $p$ 的整倍数,$(1+x)^{n\bmod p}$ 中的项的次数都 $<p$。
        
        把 $m$ 拆成 $p\lfloor m/p\rfloor + m\bmod p$,那么 $p\lfloor m/p\rfloor$ 项系数由 $(1+x^p)^{\lfloor n/p\rfloor }$ 贡献,为 $\dbinom{\lfloor n/p\rfloor}{\lfloor m/p\rfloor}$;$m\bmod p$ 项系数由 $(1+x)^{n\bmod p}$ 贡献,为 $\dbinom{n\bmod p}{m\bmod p}$。

        $\ $

        于是就得到 Lucas 定理
        $$
        \binom{n}{m}\bmod p = \binom{\lfloor n/p\rfloor}{\lfloor m/p\rfloor}\cdot\binom{n\bmod p}{m\bmod p}\bmod p
        $$
        
        \vspace{-2em}
        \hfill $\Box$

    \end{frame}

    \begin{frame}{扩展Lucas定理}
        Lucas 定理要求模数必须是质数,如果模数不是质数怎么办。
        \begin{block}{问题}
            求
            $$
            \binom{n}{m}\bmod M
            $$
            其中 $M\le 10^6$ 有质因数分解
            $$
            M = p_1^{\alpha_1}\cdots p_m^{\alpha_m}
            $$
        \end{block}
        \pause

        首先按照 $M$ 的因式分解转化成求 $\binom{n}{m}\bmod p^{\alpha}$,最后 CRT 合并即可。于是现在问题转化成求
        $$
        \binom{n}{m}\bmod p^\alpha=\frac{n!}{m!(n-m)!}\bmod p^\alpha
        $$
        其中 $p$ 是质数。
    \end{frame}

    \begin{frame}
        问题在于分母中可能含有 $p$ 这个因子,无法直接求逆,考虑把分子分母中的 $p$ 因子全部提出来。

        设 $n!,m!,(n-m)!$ 中 $p$ 因子的次数分别为 $x,y,z$(这是好求的),我们只需要求
        $$
        \frac{\frac{n!}{p^x}}{\frac{m!}{p^y}\frac{(n-m)!}{p^z}}p^{x-y-z}\bmod p^\alpha
        $$

        现在分母可以 exgcd 求逆了,我们只需要求形如下面这样的式子
        $$
        \frac{n!}{p^x}\bmod p^{\alpha}
        $$
        即 $n!$ 中去除 $p$ 因子后 $\bmod p^{\alpha}$ 的结果。
    \end{frame}
    
    \begin{frame}
        将 $n!$ 中的元素分成两个部分:
        \begin{itemize}
            \item $p$ 的倍数,这一部分的积为 $p^{\lfloor n/p\rfloor}(\lfloor n/p\rfloor)!$
            \item 非 $p$ 的倍数,这一部分又可以按照 $\bmod p^{\alpha}$ 的循环节和余项两部分
        \end{itemize}
        所以有:
$$
n! = q^{\left\lfloor\frac{n}{q}\right\rfloor} \cdot \left(\left\lfloor\frac{n}{q}\right\rfloor\right)! \cdot {\left(\prod_{i,(i,q)=1}^{q^\alpha}i\right)}^{\left\lfloor\frac{n}{q^\alpha}\right\rfloor} \cdot \left(\prod_{i,(i,q)=1}^{n\bmod q^\alpha}i\right)
$$

于是:
$$
\frac{n!}{q^{\left\lfloor\frac{n}{q}\right\rfloor}} = \left(\left\lfloor\frac{n}{q}\right\rfloor\right)! \cdot {\left(\prod_{i,(i,q)=1}^{q^\alpha}i\right)}^{\left\lfloor\frac{n}{q^\alpha}\right\rfloor} \cdot \left(\prod_{i,(i,q)=1}^{n\bmod q^\alpha}i\right)
$$

循环节内的乘积需要暴力求;$\displaystyle \left(\left\lfloor\frac{n}{q}\right\rfloor\right)!$ 可以递归求解。
\pause
\begin{block}{思考题}
    如果我要求的是一个 $\binom{n}{m}\bmod 10^{18}$ 怎么办。
\end{block}
    \end{frame}
    \subsection{例题}
    \begin{frame}{经典题}
        \begin{block}{[SDOI2010] 古代猪文}
            给定 $n,g$,求
            $$
            g^{\sum_{d|n}\binom{n}{d}}\bmod 999911659
            $$
            $1\le n,g\le 10^9$
        \end{block}
        \pause

        $999911659$ 是质数,由费马小定理,我们只需求
        $$
        \sum_{d|n}\binom{n}{d}\bmod 999911658
        $$
        将 $999911658$ 分解质因数为 $2\times 3\times 4679\times 35617$,做 exLucas 即可。
    \end{frame}

    \begin{frame}{例题}
        \begin{block}{[SHOI2015] 超能粒子炮·改}
            给定 $n,m$,求
            $$
            \sum_{k=0}^{m}\binom{n}{k}\bmod 2333
            $$
            $m\le n\le 10^9$
        \end{block}
    \end{frame}

    \begin{frame}
        
        将 $k$ 按照 $\lfloor \frac{k}{p} \rfloor$ 分类
        $$
        \begin{aligned}
        \text{原式} &= \sum_{i=0}^{\lfloor m/p \rfloor -1}\sum_{j=0}^{p-1}\binom{n}{ip+j}+\sum_{k=p\lfloor m/p \rfloor}^{m}\binom{n}{k}\\
            &=\sum_{i=0}^{\lfloor m/p \rfloor -1}\binom{\lfloor n/p\rfloor}{i}\sum_{j=0}^{p-1}\binom{n\bmod p}{j}+\sum_{k=p\lfloor m/p \rfloor}^{m}\binom{n}{k}\\
            &=2^{n\bmod p}\sum_{i=0}^{\lfloor m/p \rfloor -1}\binom{\lfloor n/p\rfloor}{i}+\sum_{k=p\lfloor m/p \rfloor}^{m}\binom{n}{k}
        \end{aligned}
        $$
        前面一项可以递归求,后面的余项可以 $O(p)$ 暴力。复杂度 $O(p\log m)$。
    \end{frame}

    \begin{frame}{例题}
        \begin{block}{[AHOI2017/HNOI2017] 抛硬币}
            给定 $a,b,k$,求满足下列条件的 $01$ 串二元组 $(s,t)$ 的数量
            \begin{itemize}
                \item $|s|=a,|t|=b$
                \item $s$ 中 $1$ 的数量严格大于 $t$ 中 $1$ 的数量
            \end{itemize}
            答案对 $10^k$ 取模。

            $1\le a,b\le 10^{15},b\le a\le b+10^4,1\le k\le 9$
        \end{block}
        \pause

        即求
        $$
        \sum_{i=0}^a\sum_{j=0}^{i-1}\binom{a}{i}\binom{b}{j}
        $$
    \end{frame}

    \begin{frame}
        $$
        \begin{aligned}
        \text{原式}&=\sum_{i=1}^a\sum_{j=0}^{a-i}\binom{a}{i+j}\binom{b}{j}=\sum_{i=1}^{a}\sum_{j=0}^b\binom{a}{i+j}\binom{b}{b-j}\\
        &=\sum_{i=1}^a\binom{a+b}{b+i}=\sum_{i=b+1}^{a+b} \binom{a+b}{i}&&\text{范德蒙德卷积}\\
        \pause
        &=\sum_{i=b+1}^{\lceil \frac{a+b}2 \rceil -1}\binom{a+b}{i}+\sum_{\lceil \frac{a+b}2 \rceil}^{a+b}\binom{a+b}{i}
        \end{aligned}
        $$
        前者只有 $\frac{a-b}{2}$ 项,可以每一项都 exLucas 求。后者是 $\sum_{i=0}^{a+b}\binom{a+b}{i}=2^{a+b}$ 的一半(如果 $a+b$ 是偶数的话还要减去一个 $\binom{a+b}{(a+b)/2}$)。

    \end{frame}

    \section{卡特兰数}
    \subsection{卡特兰数的通项公式}
    \begin{frame}{卡特兰数}
        Catalan 数列 $H_n$ 是以下问题的方案数:
        \begin{enumerate}
            \item 有一个大小为 $n\times n$ 的方格图,左下角为 $(0, 0)$ 右上角为 $(n, n)$,从左下角开始每次都只能向右或者向上走一单位,不走到对角线 $y=x$ 上方(但可以触碰)的情况下到达右上角有多少可能的路径?
            \item 在圆上选择 $2n$ 个点,将这些点成对连接起来使得所得到的 $n$ 条线段不相交的方法数?
            \item 一个栈的进栈序列为 $1,2,3, \cdots ,n$ 有多少个不同的出栈序列?
            \item $n$ 个结点可构造多少个不同的二叉树?
            \item $n$ 对括号能组成的括号序列数?
            \item ……
        \end{enumerate}
        \pause
        
        众所周知,卡特兰数有以下两个表达式
        $$
        \begin{aligned}
            H_n&=\begin{cases}
            \sum_{i=0}^{n-1}H_iH_{n-i-1}&(n\ge 2)\\
            1&(n=0,1)
            \end{cases}\\
        &=\binom{2n}{n}-\binom{2n}{n-1}
        \end{aligned}
        $$

    \end{frame}

    \begin{frame}{卡特兰数的通项公式}
        \begin{block}{卡特兰数的通项公式}
            \vspace{-1em}
            $$
            H_n = \frac{\binom{2n}{n}}{n+1}
            $$
            \vspace{-1em}
        \end{block}
        由此还可以得到一个递推式
        $$
        H_n=\frac{H_{n-1}(4n-2)}{n+1}
        $$
    \end{frame}

    \begin{frame}{卡特兰数通项公式的证明}
        \vspace{-1em}
$$
H_n=\sum_{i=0}^{n-1}H_{i}H_{n-i-1} \quad (n\ge 2)
$$
其中 $H_0=1,H_1=1$。设它的普通生成函数为 $H(x)$,利用卷积,得到它的一个方程。

$$
\begin{aligned}
H(x)&=\sum_{n\ge 0}H_nx^n\\
&=1+\sum_{n\ge 1}\sum_{i=0}^{n-1}H_ix^iH_{n-i-1}x^{n-i-1}x\\
&=1+x\sum_{i\ge 0}H_{i}x^i\sum_{n\ge 0}H_{n}x^{n}\\
&=1+xH^2(x)
\end{aligned}
$$

解得
$$
H(x)=\frac{1\pm \sqrt{1-4x}}{2x}
$$
\end{frame}

\begin{frame}
    
那么这就产生了一个问题:我们应该取哪一个根呢?我们将其分子有理化:
$$
H(x)=\frac{2}{1\mp \sqrt{1-4x}}
$$
代入 $x=0$,我们得到的是 $H(x)$ 的常数项,也就是 $H_0$。当 
 
$H(x)=\dfrac{2}{1+\sqrt{1-4x}}$ 的时候有 $H(0)=1$,满足要求。而另一个解会出现分母为 $0$ 的情况(不收敛),舍弃。

因此我们得到了卡特兰数生成函数的封闭形式:

$$
H(x)=\frac{1- \sqrt{1-4x}}{2x}
$$

\end{frame}

\begin{frame}
    
接下来我们要将其展开。使用牛顿二项式定理。

$$
\begin{aligned}
(1-4x)^{\frac{1}{2}}
&=\sum_{n\ge 0}\binom{\frac{1}{2}}{n}(-4x)^n\\
&=1+\sum_{n\ge 1}\frac{\left(\frac{1}{2}\right)^{\underline{n}}}{n!}(-4x)^n
\end{aligned}
$$

其中
$$
\begin{aligned}
\left(\frac{1}{2}\right)^{\underline{n}}
&=\frac{1}{2}\frac{-1}{2}\frac{-3}{2}\cdots\frac{-(2n-3)}{2}\\
&=\frac{(-1)^{n-1}(2n-3)!!}{2^n}\\
&=\frac{(-1)^{n-1}(2n-2)!}{2^n(2n-2)!!}\\
&=\frac{(-1)^{n-1}(2n-2)!}{2^{2n-1}(n-1)!}
\end{aligned}
$$
\end{frame}

\begin{frame}
于是
$$
\begin{aligned}
(1-4x)^{\frac{1}{2}}
&=1+\sum_{n\ge 1}\frac{(-1)^{n-1}(2n-2)!}{2^{2n-1}(n-1)!n!}(-4x)^n\\
&=1-\sum_{n\ge 1}\frac{(2n-2)!}{(n-1)!n!}2x^n
=1-\sum_{n\ge 1}\binom{2n-1}{n}\frac{1}{(2n-1)}2x^n
\end{aligned}
$$
带回原式得到

$$
\begin{aligned}
H(x)&=\frac{1- \sqrt{1-4x}}{2x}=\frac{1}{2x}\sum_{n\ge 1}\binom{2n-1}{n}\frac{1}{(2n-1)}2x^n\\
&=\sum_{n\ge 1}\binom{2n-1}{n}\frac{1}{(2n-1)}x^{n-1}=\sum_{n\ge 0}\binom{2n+1}{n+1}\frac{1}{(2n+1)}x^{n}\\
&=\sum_{n\ge 0}\binom{2n}{n}\frac{1}{n+1}x^{n}\\
\end{aligned}
$$
这样我们就得到了卡特兰数的通项公式。
\end{frame}

\begin{frame}{$\binom{2n}{n}$ 的生成函数}
    \begin{itemize}
        \item 
        利用 $H_n=\dbinom{2n}{n}\dfrac{1}{n+1}$ 的生成函数是 $H(x)=\dfrac{1- \sqrt{1-4x}}{2x}$,求 $\dbinom{2n}{n}$ 的生成函数。
    \end{itemize}
    \pause

    $$
    Q(x)=\sum_{n\ge 0} \binom{2n}{n}x^n=(x\cdot H)'(x) = \frac{1}{\sqrt{1-4x}}
    $$
\end{frame}

\subsection{例题}
\begin{frame}{例题}
    \begin{block}{[THUSC2021] 种树}
        这是一道\textbf{通信题}

给你一棵 $n$ 个点的树,你需要完成一个编码器和解码器。

编码器需要返回一个$128$位二进制数,用于表示这棵树。解码器需要根据编码器返回的 $128$ 位二进制数还原这棵树(还原出一棵同构的树即可)。两棵树同构当且仅当它们在节点无标号、考虑孩子顺序的意义下相同。



$n\le 70,T\le 10^5$
    \end{block}

    \pause

    可以先考虑 $n\le 65$。

    \pause

    递归地表示这棵树:如果往儿子走,则当前位填$1$,往父亲走填$0$。还原时也递归地还原这棵树即可。然后因为$1$一定往儿子走,不会往父亲走,前后两位可以去掉,可以将长度控制在$2n-2$以内。
\end{frame}

\begin{frame}
    上述过程是一个括号匹配的过程,填$1$则为左括号,填$0$为右括号。

    而我们有可以省去最靠前的一个$($和最靠后的一个$)$,这样的合法括号序列个数为$Catalan(n-1)$个。

    而$Catalan(69)$刚好比$2^{128}-1$小,也就是我们现在要将一个括号序列与一个数字一一对应。

    \pause

    我们计算这个括号序列的字典序是所有合法括号序列的第几项,也就是对于每一个填$1$的位,求出这一位填$0$,前面所有位都一样的合法括号序列数,可以用组合数计算。这样我们就完成了编码。

    解码时就是上述过程的逆过程。如果当前位填$0$之后,合法括号序列的个数小于我们所需的,那么这一位必须填$1$,否则必须填$0$。这样我们完成了解码。
\end{frame}

\begin{frame}{例题}
    \begin{block}{某联考题}
        给定一个长度为 $n$ 的括号序列 $c$(不一定合法,但左括号和右括号数量相同)

    定义一个 $1,2,\cdots,n$ 的排列 $p$ 是好的,当且仅当括号序列 $d(\forall i\in [1,n],d_i=c_{p_i})$ 是合法的括号序列,定义 $c$ 的价值为所有好的排列的逆序数总和。

    $c$ 并不稳定,会发生 $q$ 次改变,具体来说,$c$ 中的某两个元素会发生交换。(改变是永久的)

    你需要对一开始的 $c$ 和每次改变后的 $c$ 分别求出它的价值对输入的质数 $P$ 取模的结果。

    $n\le 10^7,q\le 10^5$
    \end{block}


\end{frame}

    \begin{frame}
        记$m=\frac{n}{2}$,一个排列将原序列 $c$ 映射为一个合法括号序列 $d$。

以下四种映射的其中之一会造成 $1$ 的贡献(颜色用于表示相对顺序)

\begin{itemize}
    \item 子序列 ${\color{red}(}{\color{blue}(}\to{\color{blue}(}{\color{red}(}$
    \item 子序列 ${\color{red})}{\color{blue})}\to{\color{blue})}{\color{red})}$
    \item 子序列 ${\color{red}(}{\color{blue})}\to{\color{blue})}{\color{red}(}$
    \item 子序列 ${\color{red})}{\color{blue}(}\to{\color{blue}(}{\color{red})}$
\end{itemize}


记原序列中 \texttt{((},\texttt{))},\texttt{()},\texttt{)(} 子序列的个数分别为$a_0,a_1,a_2,a_3$,对于每个 \texttt{((},\texttt{))},\texttt{()},\texttt{)(} 子序列,分别有 $b_0,b_1,b_2,b_3$ 个排列的映射(对于不同位置的子序列,这个排列的数量应当是相同的)会造成 $1$ 的贡献,那么最终答案为
$$
a_0b_0+a_1b_1+a_2b_2+a_3b_3
$$
\pause
其中 $a_0,a_1,a_2,a_3$ 是好维护的:

现在考虑交换两个位置
\begin{itemize}
    \item 如果这两个位置的字符一样,显然对答案没有变化
    \item 否则$a_2,a_3$的变化量为$\pm (r-l)$
\end{itemize}

$b_0,b_1,b_2,b_3$ 应该是只与 $n$ 有关的常数,我们下面考虑求出 $b_0,b_1,b_2,b_3$。
    \end{frame}

    \begin{frame}
        \begin{itemize}
            \item 求 $b_0,b_1$
        \end{itemize}

        二者的值是相同的,所以就求 $b_0$ 即可。

        对于所有 $Catalan(m)$ 个括号序列,这些序列里都有 $\frac{m(m-1)}{2}$ 个 \texttt{((} 子序列。在这 $\frac{m(m-1)}{2}$ 个 \texttt{((} 子序列中任选一个与原序列的 \texttt{((} 匹配。因为我们要求产生贡献必须交换相对位置,所以这两个位置的映射方式就固定了。

        剩下 $m-2$ 个 \texttt{(} 和 $m$ 个 \texttt{)} 的映射方式是任意的,方案数为 $(m-2)!\cdot m!$。

        总方案数为
        $$
        b_0=b_1 = Catalan(m)\cdot \frac{m(m-1)}{2} \cdot (m-2)!\cdot m!= \frac 12Catalan(m) (m!)^2
        $$
    \end{frame}

    \begin{frame}
        \begin{itemize}
            \item 求 $b_2,b_3$
        \end{itemize}

        对于所有 $Catalan(m)$ 个括号序列,只要这个序列中有一个 \texttt{)(} 子序列,这个子序列和原序列的 \texttt{()} 对应上就会产生 $1$ 的贡献,此时剩下的位置的映射方式是任意的,方案数为 $((m-1)!)^2$,即
        $$
        b_2 = \left(\sum_{\text{无标号括号序列}d} d\text{ 中)(子序列数量}\right) ((m-1)!)^2
        $$

        同理
        $$
        b_3 = \left(\sum_{\text{无标号括号序列}d} d\text{ 中()子序列数量}\right) ((m-1)!)^2
        $$

        一个括号序列中 \texttt{()} 和 \texttt{)(} 子序列的数量之和是 $m^2$ 个,于是两式相加得
        $$
        b_2+b_3 = Catalan(m)m^2((m-1)!)^2=Catalen(m)(m!)^2
        $$
        于是只需求出 $b_2$ 即可求出 $b_3$。
    \end{frame}
    \begin{frame}
        考虑求所有合法无标号括号序列中 \texttt{)(} 子序列的数量。如果两对括号相离,则贡献为$1$,否则没有贡献。
        
        直接做不好做,容斥一下,转为求两对括号相包含的贡献为 $1$。枚举这一对括号包含了 $i$ 对括号,包含的括号内部可以随意排列($Catalan(i)$),剩下的也可以随便排列($Catalan(m-i-1)$),然后将自己和自己包含的括号随便插入另外$m-i-1$个括号构成的括号序列中,共 $2(m-i-1)+1$ 个空。于是,相互包含的括号对数为
$$
\sum_{i=1}^{m-1}(2(m-i-1)+1)iCatalan(i)Catalan(m-i-1)
$$

\vspace{-1em}
而在所有括号序列中任选两对匹配括号的方案是$Catalan(m)\times \dfrac{m(m-1)}{2}$,所以 \texttt{)(} 子序列出现的次数是
$$
Catalan(m)\times \dfrac{m(m-1)}{2}-\sum_{i=1}^{m-1}(2(m-i-1)+1)iCatalan(i)Catalan(m-i-1)
$$

这个数再乘上$((m-1)!)^2$就是$b_2$。用 $Catalen(m)(m!)^2-b_2$ 就是 $b_3$。

总复杂度 $O(n+q)$。
    \end{frame}

    \section{二项式反演}
    \subsection{二项式反演}
    \begin{frame}{容斥}
        记 $f_n$ 表示恰好使用 $n$ 个不同元素形成特定结构的方案数,$g_n$ 表示从 $n$ 个不同元素中选出若干个元素形成特定结构的总方案数。

若已知 $f_n$ 求 $g_n$,那么显然有:

$$
g_n = \sum_{i = 0}^{n} \binom{n}{i} f_i
$$

若已知 $g_n$ 求 $f_n$,使用容斥,我们得到:
$$
f_n = \sum_{i = 0}^{n} \binom{n}{i} (-1)^{n-i} g_i
$$

    \end{frame}

    \begin{frame}{二项式反演}
        上面容斥的式子本质上就是二项式反演
        \begin{block}{二项式反演}
            $$
            g_n = \sum_{i = 0}^{n} \binom{n}{i} f_i\iff f_n = \sum_{i = 0}^{n} \binom{n}{i} (-1)^{n-i} g_i
            $$
        \end{block}
        它还有一个等价形式
        
        \begin{block}{二项式反演的一个等价形式}
            $$
            g_n = \sum_{i = 0}^{n}(-1)^i\binom{n}{i} f_i\iff f_n = \sum_{i = 0}^{n} \binom{n}{i} (-1)^{i} g_i
            $$
        \end{block}
    \end{frame}
    
    \begin{frame}{二项式反演的证明}
        我们只证明 $ g_n = \sum_{i = 0}^{n} \binom{n}{i} f_i\implies f_n = \sum_{i = 0}^{n} \binom{n}{i} (-1)^{n-i} g_i$,另一边是同样的。
        $$
        \begin{aligned}
            \sum_{i = 0}^{n} \binom{n}{i} (-1)^{n-i} g_i&= \sum_{i = 0}^{n} \binom{n}{i} (-1)^{n-i} \sum_{j=0}^i\binom{i}{j}f_j\\
            &=  \sum_{j=0}^nf_j\sum_{i=j}^n\binom{n}{i}\binom{i}j(-1)^{n-i}\\
            \pause
            &= \sum_{j=0}^nf_j\sum_{i=j}^n\binom{n}{j}\binom{n-j}{i-j}(-1)^{n-i}&&\text{三项式版恒等式}\\
            &=\sum_{j=0}^n\binom{n}{j}f_j\sum_{k=0}^{n-j}\binom{n-j}{k}(-1)^{n-j-k}&&\text{令 }k=i-j\\
            &=\sum_{j=0}^n\binom{n}{j}f_j[n-j=0]=f_n && \text{二项式定理}
        \end{aligned}
        $$

        \vspace{-3em}
        \hfill $\Box$
    \end{frame}

    \begin{frame}{矩阵形式}
        如果把 $f_0,f_1,\cdots, f_n$ 和 $g_0,,g_1,\cdots, g_n$ 写成列向量,则二项式反演可以写为

        \begin{gather*}
        \underbrace{
        \begin{pmatrix}
        \binom{0}{0}&0&\cdots&0\\
        \binom{1}{0}&\binom{1}{1}&\cdots&0\\
        \vdots&\vdots&\ddots&\vdots\\
        \binom{n}{0}&\binom{n}{1}&\cdots&\binom{n}{n}
        \end{pmatrix}
        }_{\boldsymbol A}
        \begin{pmatrix}
            f_0\\
            f_1\\
            \vdots\\
            f_n
        \end{pmatrix}=\begin{pmatrix}
            g_0\\
            g_1\\
            \vdots\\
            g_n
        \end{pmatrix}
        \iff\\
        \underbrace{
        \begin{pmatrix}
        \binom{0}{0}&0&\cdots&0\\
        -\binom{1}{0}&\binom{1}{1}&\cdots&0\\
        \vdots&\vdots&\ddots&\vdots\\
        (-1)^{n}\binom{n}{0}&(-1)^{n-1}\binom{n}{1}&\cdots&\binom{n}{n}
        \end{pmatrix}
        }_{\boldsymbol B}
        \begin{pmatrix}
            g_0\\
            g_1\\
            \vdots\\
            g_n
        \end{pmatrix}=\begin{pmatrix}
            f_0\\
            f_1\\
            \vdots\\
            f_n
        \end{pmatrix}
    \end{gather*}

        刚才给出的证明等价于证明 $\boldsymbol A \cdot \boldsymbol B=\boldsymbol I$。
    \end{frame}

    \begin{frame}
        将 $\boldsymbol A$ 和 $\boldsymbol B$ 转置之后得到的矩阵仍然是互逆的,所以,我们还可以得到二项式反演的另一个形式

        \begin{block}{二项式反演的另一个形式}
            $$
            g_m = \sum_{i=m}^n\binom{i}{m}f_i\iff f_m=\sum_{i=m}^n(-1)^{i-m}\binom{i}{m}g_i
            $$
        \end{block}

        从组合意义来说,前面的形式中,$g_m$ 表示“至多”$m$ 个的方案数,而在这种形式中,$g_m$ 表示“至少”$m$ 个的方案数。

        二项式反演最常见的转化,就是将“恰好”转化为“至多”或者“至少”, 再二项式反演回来。
    \end{frame}
    \subsection{例题}
    \begin{frame}{例题}
        \begin{block}{洛谷P10596 BZOJ2839 集合计数}
        一个有 $n$ 个元素的集合有 $2^n$ 个不同子集(包含空集),现在要在这 $2^n$ 个集合中取出若干集合(至少一个),使得它们的交集的元素个数恰好为 $k$,求取法的方案数,答案模 $10^9+7$。

        $1\leq n\leq 10^6$,$0\leq k\leq n$。
        \end{block}

        \pause

        令 $f_k$ 表示交集个数“\textbf{至少}”是 $k$ 的方案数。那么我们先在 $n$ 个元素中选 $k$ 个作为交集的部分。包含这 $k$ 的元素的集合有 $2^{n-k}$ 个,在这些集合中任选至少一个,方案数为 $2^{2^{n-k}}-1$,则
        $$
        f_k=\binom{n}{k}(2^{2^{n-k}}-1)
        $$
        注意,这里“至少 $k$ 个”的含义是,钦定了满足性质的 $k$ 个元素,剩下的性质不作限制。也就是说,一个实际上交集大小为 $m$ 的方案,它在 $f_k$ 中被计数了 $\binom{m}{k}$ 次。
    \end{frame}
    \begin{frame}
        设 $g_k$ 表示交集个数\textbf{恰好}是 $k$ 的方案数。枚举实际的交集大小 $i$,在这 $i$ 个元素里面任选 $k$ 个作为满足 $f_k$ 限制的那 $k$ 个元素,我们有关系式
        $$
        f_k=\sum_{i=k}^n\binom{i}{k}g_i
        $$
        \pause

        由二项式反演
        $$
        g_k = \sum_{i=k}^n(-1)^{i-k}\binom{i}{k}f_i=\sum_{i=k}^n(-1)^{i-k}\binom{i}{k}\binom{n}{i}(2^{2^{n-i}}-1)
        $$
        复杂度 $O(n)$。
    \end{frame}

    \begin{frame}{例题}
        \begin{block}{洛谷P4859 已经没有什么好害怕的了}
            给定 $n$ 个 $A$ 类物品与 $n$ 个 $B$ 类物品,每个物品具有一个权值,将 $A$ 类物品与 $B$ 类物品两两配对,使得 $A$ 物品权值 $>$ $B$ 物品的组恰好有 $k$ 个。求配对方案数 $\bmod 10^9+9$。
            
            所有权值互不相同。$n\le 2000$
        \end{block}
        \pause

        题目的要求是恰好 $k$ 个,比较难求,但我们可以先钦定 $k$ 组,让这 $k$ 组满足 $A > B$,其余组随便选择,求出答案后二项式反演回来。

        考虑 $f_{i,j}$ 表示已经配对了前 $i$ 个 $A$,钦定了 $j$ 组满足 $A > B$,剩下的不管,将 $A$ 从小到大排序, 那么得到转移方程:
        $$
        f_{i,j} = f_{i−1,j} + f_{i−1,j−1}(d_i − (j − 1))
        $$
        其中 $d_i$ 表示 $< a_i$ 的 $b$ 的数量。那么此时我们得到 $g_i = (n − i)!f_{n,i}$,表示钦定了 $i$ 组 $A > B$ 的方案数

        于是直接用上面的柿子反演回来即可。复杂度 $O(n^2)$。

    \end{frame}

    \begin{frame}{例题}
        \begin{block}{洛谷P6295 有标号 DAG 计数}
            求 $n$ 个点的有标号弱连通DAG数量 $\bmod 998244353$。

            $n,T\le 10^5$
        \end{block}
        \pause

        $\exp$ 组合意义为:有标号对象组成的集合个数。设 $n$ 个点的有标号DAG(不一定弱连通)数量的生成函数为 $G(x)$,$n$ 个点的有标号弱连通DAG数量的生成函数为 $F(x)$,则
        $$
        G(x)=\exp(F(x))\implies F(x)=\ln G(x)
        $$
        我们要求的是 $F(x)$,所以只需求出 $G(x)$ 再取 $\ln$ 即可。
    \end{frame}

    \begin{frame}
        设 $g_i$ 表示 $i$ 个点的有标号 DAG 数量,即 $G(x)$ 是 $g_i$ 的生成函数。

        令 $h_{i,j}$ 表示 $i$ 个点的图,钦定其中 $j$ 个点并让这 $j$ 个点入度为 $0$ 的方案数(类似于之前的“至少 $j$ 个”);令 $c_{i,j}$ 表示 $i$ 个点的图恰好有 $j$ 个点入度为 $0$ 的方案数,则有二项式反演:

        $$
        h_{i,j}=\sum_{k=j}^i\binom{k}{j}c_{i,k}\iff c_{i,j}=\sum_{k=j}^i\binom{k}{j}(-1)^{k-j}h_{i,k}
        $$
        \pause

        同时我们还有
        $$
        h_{i,j} = \binom{i}{j}2^{j(i-j)}g_{i-j}
        $$
        $$
        g_i=\sum_{j=1}^ic_{i,j}
        $$
    \end{frame}

    \begin{frame}
        代入,得
        $$
        \begin{aligned}
        g_n&=\sum_{i=1}^nc_{n,i}=\sum_{i=1}^n\sum_{j=i}^n\binom{j}{i}(-1)^{j-i}h_{n,j}\\
        &=\sum_{i=1}^n\sum_{j=i}^n\binom{j}{i}(-1)^{j-i}\binom{n}{j}2^{j(n-j)}g_{n-j}\\
        &=\sum_{j=1}^n\binom{n}{j}2^{j(n-j)}g_{n-j}\sum_{i=1}^j\binom{j}{i}(-1)^{j-i}\\
        &=\sum_{j=1}^n(-1)^{j+1}\binom{n}{j}2^{j(n-j)}g_{n-j}
        \end{aligned}
        $$
        \pause
        将 $j(n-j)$ 拆成 $\binom{n}{2}-\binom{j}{2}-\binom{n-j}{2}$,把上式写成卷积的形式

        (或者把 $2^{j(n-j)}$ 写成 $\sqrt{2}^{n^2-j^2-(n-j)^2}$,这就需要求出 $\sqrt 2$ 的二次剩余)
    \end{frame}

    \begin{frame}
        $$
        \begin{aligned}
        g_n&=\sum_{j=1}^n(-1)^{j+1}\binom{n}{j}2^{j(n-j)}g_{n-j}\\
        \frac{g_n}{n!2^{\binom{n}{2}}}&=\sum_{j=1}^n\frac{(-1)^{j+1}}{2^{\binom j2}j!}\frac{g_{n-j}}{2^{\binom {n-j}2}(n-j)!}
        \end{aligned}
        $$

        令
        $$
        \begin{aligned}
        P(x)&=\sum_{i=0}\frac{g_i}{2^{\binom{i}{2}}i!}x^i\\
        Q(x)&=\sum_{i=1}\frac{(-1)^{i+1}}{2^{\binom i2}i!}x^i
        \end{aligned}
        $$

        那么 $P(x)=P(x)Q(x)+1$,解得
        $$
        P(x)=\frac{1}{1-Q(x)}
        $$
        于是做一遍多项式求逆求出 $P$,然后就得到了 $g_n$,即 $G(x)$,再取 $\ln$ 得到 $F$。复杂度 $O(n\log n)$。
    \end{frame}

    \begin{frame}{例题}
        \begin{block}{[CTS2019] 珍珠}
            称一个长度为 $n$,元素取值 $[1,D]$ 的整数序列是合法的,当且仅当其中能够选出至少 $m$ 对相同元素(不能重复选出元素)。

            问合法序列个数 $\bmod 998244353$。

            $1\le m\le 10^9,1\le n\le 10^9,1\le D\le 10^5$
        \end{block}
        
        \pause

        对于一个序列,设其中 $i$ 的出现次数为 $c_i$,则题目的限制可以写为
        $$
        \sum_{i=1}^D\left\lfloor\frac{c_i}{2}\right\rfloor\ge m\implies \sum_{i=1}^n\frac{c_i-(c_i\bmod 2)}{2}\ge m
        $$
        其中 $\sum_{i=1}^D c_i=n$,则
        $$
        \sum_{i=1}^Dc_i\bmod 2\le n-2m
        $$

        其中 $\sum_{i=1}^Dc_i\bmod 2$ 可以理解为出现次数为奇数的元素数量。
    \end{frame}

    \begin{frame}
        设 $f_i$ 表示恰好有 $i$ 种元素出现了奇数次的方案数,那么:
        $$
        ans=\sum_{i=0}^{\min(D,n-2m)}f_i
        $$
        \pause
        先求出 $g_i$ 表示钦定其中 $i$ 种元素出现了奇数次,其他元素随便取(“至少”)的方案数。那么:
        $$
        g_i=\sum_{j=i}^{D}f_i\binom{i}{j}\Rightarrow f_i=\sum_{j=i}^{D}g_j\binom{i}{j}(-1)^{i-j}
        $$
        因此如果求得 $g$,可以直接 $\mathcal O(D\log D)$ 卷积得到 $f$。
    \end{frame}

    \begin{frame}
        因为序列中需要考虑顺序,所以考虑 EGF,每个颜色贡献的 EGF 之积的 $[x^n]$ 会贡献给 $g_k$。
        
        \begin{itemize}
            \item 被钦定的那 $k$ 个颜色,只在选择奇数次时作出贡献,因此其 OGF 为 $x+x^3+x^5+\cdots$,EGF为 $\frac{e^x-e^{-x}}{2}$
            \item 剩下的 $n-k$ 个颜色,贡献都是 $1$,EGF 就是 $e^x$
        \end{itemize}

        于是
        $$
        \begin{aligned}
            g_k&=\binom{D}{k}n![x^n](\frac{e^x-e^{-x}}{2})^k(e^x)^{D-k}\\
            &=\binom{D}{k}\frac{n!}{2^k}[x^n](e^x-e^{-x})^k(e^x)^{D-k}\\
            &=\binom{D}{k}\frac{n!}{2^k}[x^n]\sum_{j=0}^{k}\binom{k}{j}(e^x)^j(-e^{-x})^{k-j}(e^x)^{D-k}\\
            &=\binom{D}{k}\frac{n!}{2^k}\sum_{j=0}^{k}\binom{k}{j}(-1)^{k-j}[x^n](e^x)^{D-2(k-j)}\\
        \end{aligned}
        $$
        其中 $[x^n](e^x)^{D-2(k-j)} = \frac{(D-2(k-j))^n}{n!}$
        
    \end{frame}

    \begin{frame}
        代入,得
        $$
        \begin{aligned}
            g_k&=\binom{D}{k}\frac{1}{2^k}\sum_{j=0}^{k}\binom{k}{j}(-1)^{k-j}[D-2(k-j)]^{n}\\
            &=\binom{D}{k}\frac{k!}{2^k}\sum_{j=0}^{k}\frac{(-1)^{k-j}[D-2(k-j)]^{n}}{j!(k-j)!}\\
            \end{aligned}
        $$

        上式也是一个卷积的形式,具体地
        $$
        a_i=\frac{1}{i!},b_i=\frac{(-1)^{i}(D-i)^n}{i!}
        $$
        两者卷积得到 $g$,再用二项式反演的式子卷积得到 $f$,复杂度 $O(D\log D)$。
    \end{frame}

    \section{牛顿级数}
    \subsection{牛顿级数}
    \begin{frame}{高阶差分}
        定义差分算子:
        $$
        \Delta f(n)=f(n+1)-f(n)
        $$
        以及它的复合
        $$
        \Delta^{m+1}f(n)=\Delta^m f(n+1)-\Delta^m f(n)
        $$
        \pause
        它对于加法、数乘和乘法有如下运算律:
        \begin{gather*}
        \Delta(f(n)+g(n))=\Delta f(n)+\Delta g(n)\\
        \Delta (c(f(n)))=c\cdot \Delta f(n)\\
        \Delta(f(n)g(n)) = f(n)\Delta g(n)+g(n+1)\Delta f(n)
        \end{gather*}
    \end{frame}

    \begin{frame}{$n$ 阶差分的性质}
        \begin{block}{$n$ 阶差分的性质}
            \vspace{-1em}
            $$
            \Delta ^n f(x)=\sum_{k=0}^n\binom{n}{k}(-1)^{n-k}f(x+k)
            $$
            \vspace{-1em}
        \end{block}
        \pause
        \textbf{证明}:定义平移算子 $\operatorname{E} f(x) = f(x+1)$,从而 $\Delta=\operatorname{E}-1$,于是根据二项式定理
        $$
        \Delta^n = (\operatorname E-1)^n=\sum_{k=0}^n\binom{n}{k}E^k(-1)^{n-k}=\sum_{k=0}^n\binom{n}{k}(-1)^{n-k}f(x+k)
        $$

        \vspace{-3em}
        \hfill $\Box$
        \pause

        \vspace{1.5em}
        \begin{block}{组合数的差分}
            计算
            \vspace{-1em}
            $$
            \Delta^m \left(\binom{x}{k}\right)
            $$
            \vspace{-1em}
        \end{block}
        \pause
        加法公式给出,$\displaystyle \Delta \left(\binom{x}{k}\right)=\binom{x}{k-1}$,于是 $\displaystyle \Delta^m \left(\binom{x}{k}\right)=\binom{x}{k-m}.$
    \end{frame}

    \begin{frame}{牛顿级数}
        因为 $\binom{x}{n}$(或者说 $x^{\underline n}$) 是一个 $n$ 次多项式,所以一个多项式 $f(x)=a_0+a_1x+a_2x^2+\cdots+a_nx^n$ 一定可以写为如下形式
        $$
        f(x)=c_0\binom{x}{0}+c_1\binom{x}{1}+c_2\binom{x}{2}+\cdots+c_n\binom{x}{n}
        $$
        这样一个展开式称为 $f(x)$ 的\textbf{牛顿级数}。

        \pause
        取 $\Delta^k f(0)$,
        $$
        \Delta^kf(0) = \sum_{i=0}^nc_i\binom{0}{i-k} = c_k
        $$

        所以一个多项式可以很容易地写成牛顿级数的形式
        $$
        f(x)=f(0)\binom{x}{0}+\Delta f(0)\binom{x}{1}+\Delta^2 f(0)\binom{x}{2}+\cdots+\Delta^n f(0)\binom{x}{n}
        $$
    \end{frame}

    \begin{frame}{例子}
        \begin{block}{求下列高阶差分}
            \vspace{-1em}
            $$
            \Delta^n x^n,\ \Delta^m x^n(m>n)
            $$
            \vspace{-1.5em}
        \end{block}
        将 $x^n$ 写成牛顿级数
        \begin{gather*}
        f(x)=c_0\binom{x}{0}+c_1\binom{x}{1}+c_2\binom{x}{2}+\cdots+c_n\binom{x}{n}\\
        \Delta^n f(x) = c_n\binom{x}{0} = c_n\\
        \Delta^m f(x) = 0
        \end{gather*}

        而 $c_n\binom{x}{n}=\frac{c_n}{n!}x^{\underline n}$ 中的最高次项就是 $x^n$,于是
        $$
        c_n=n!,\quad \Delta^n f(x) = n!
        $$
    \end{frame}

    \begin{frame}
        设 $n$ 为多项式 $f(x)=a_0+a_1x+a_2x^2+\cdots+a_nx^n$ 的次数,对 $x=0$ 利用 
        $$
        \Delta ^n f(x)=\sum_{k=0}^n\binom{n}{k}(-1)^{n-k}f(x+k)
        $$
        和
        $$
        \Delta^n f(0)=c_n = n! a_n
        $$
        我们可以得到下面这个恒等式
        $$
        \sum_{k=0}^n\binom{n}{k}(-1)^{k}(a_0+a_1k+a_2k^2+\cdots+a_nk^n)=(-1)^nn!a_n
        $$
    \end{frame}

    \begin{frame}{应用}
        作为应用
        \begin{block}{求下列式子的封闭形式}
            \begin{enumerate}
                \item $\displaystyle \sum_{k}(-1)^k\binom{n}{k}(n-k)^n$
                \item $\displaystyle \sum_{k}(-1)^k\binom{n}{k}\binom{kn}{n}$
            \end{enumerate}
        \end{block}
        \pause
        \begin{block}{答案}
            \begin{enumerate}
                \item $n!$
                \item $(-1)^nn^n$
            \end{enumerate}
        \end{block}
    \end{frame}

    \subsection{例题}
    \begin{frame}{例题}
        \begin{block}{拉格朗日插值}
            但是请使用牛顿级数。

            具体地,已知 $n$ 阶多项式 $P(x)$ 在前 $n + 1$ 个点处的值
            $P(0), P(1),\cdots, P(n)$,求 $P(x)$ 的值,要求 $O(n)$。
        \end{block}
        \pause

        $$
        \begin{aligned}
            P(x)= & \sum_{k\leq n}c_k\binom xk=\sum_{k\leq n}\binom xk\sum_{t}(-1)^{k-t}\binom kt P(t)\\
                = & \sum_{t\leq n}P(t)\sum_{t\leq k\leq n}\binom xk(-1)^{k-t}\binom kt\\
                = & \sum_{t\leq n}P(t)\binom xt\sum_{t\leq k\leq n}\binom {x-t}{k-t}(-1)^{k-t}
        \end{aligned}
        $$
    \end{frame}

    \begin{frame}
        $$
        \begin{aligned}
            = & \sum_{t\leq n}P(t)\binom xt\sum_{k\leq n-t}\binom{x-t}k(-1)^k\\
            = & \sum_{t\leq n}P(t)\binom xt\binom{x-t-1}{n-t}(-1)^{n-t}&&\text{交错和}\\
            = & \sum_{t\leq n}(-1)^{n-t}P(t)\frac{x^{\underline{n+1}}}{t!(n-t)!(x-t)}
        \end{aligned}
        $$

        预处理前缀积和后缀积即可做到 $O(n)$。
    \end{frame}

    \begin{frame}{例题}
        \begin{block}{洛谷P7438 更简单的排列计数}

        设 $\operatorname{cyc}_\pi$ 将长为 $n$ 的排列 $\pi$ 当成置换时所能分解成的循环个数。给定两个整数 $n,k$ 和一个 $k-1$ 次多项式,对 $1\leq m\leq n$ 求:

        $$
        \sum\limits_{\pi}F(\operatorname{cyc}_{\pi})\bmod 998244353
        $$

        其中 $\pi$ 是长度为 $m$ 且不存在位置 $i$ 使得 $\pi_i=i$ 的排列。


        $1\leq n\leq 6\times 10^5$,$1\leq k\leq 100$,$0\leq [x^k]F(x)\leq 998244352$。
        \end{block}
        \pause

        
        把多项式 $F(x)$ $O(k^2)$ 转成牛顿级数,
        $$
        F(x)=\sum_{i=0}^{k-1}a_i\binom{x}{i}
        $$
        现在问题变为求出答案的每一项,即对每一个 $1\le m\le n$ 和 $0\le i< k$ 求出 $\sum_{\pi,|\pi|=m}\binom{\operatorname{cyc}_{\pi}}{i}$。
    \end{frame}

    \begin{frame}
        错排数的 EGF 为许多长度大于 1 的环构成的大于 1 的排列的 $\exp$。其中环的 EGF 为
        $$
        \sum_{i=2} \frac{(i-1)!}{i!}x^i = -\ln (1-x)-x
        $$
        所以错排数的 EGF 为
        $$
        G(x)=e^{-\ln (1-x)-x}
        $$

        \pause
        这题除了要求错排的个数,还要求出这些错排中选 $i$ 个环的方案数。为此,引入一个新变量 $y$,给每个环的EGF都带上$(1+y)$ 的因子,变为
        $$
        (-\ln (1-x)-x)(1+y)
        $$
        含这个 $y$ 因子表示选这个环,不含这个 $y$ 因子表示不选这个环
        $$
        G(x,y)=e^{(-\ln (1-x)-x)(1+y)}
        $$
        我们所求的 $\sum_{\pi,|\pi|=m}\binom{\operatorname{cyc}_{\pi}}{i}$ 就是 $[x^my^i]G$。
    \end{frame}

    \begin{frame}
        记 $g_{a,b} = [x^ay^b]G$。将 $G$ 对 $x$ 求导
        $$
        \frac{\partial G}{\partial x}(x,y) = \frac{x(1+y)}{1-x}G
        $$
        那么
        $$
        \begin{aligned}
        (n+1)g_{n+1,k} &= \sum_{i=0}^{n-1}(g_{i,k}+g_{i,k-1})\\
        ng_{n,k}&=\sum_{i=0}^{n-2}(g_{i,k}+g_{i,k-1})
        \end{aligned}
        $$
        于是我们得到了递推式
        $$
        (n+1)g_{n+1,k} = ng_{n,k} + (g_{n-1,k}+g_{n-1,k-1})
        $$
        $O(nk)$ 递推即可求出 $g$。总复杂度 $O(nk+k^2)$。
    \end{frame}

    \section{完}
	\begin{frame}
		\begin{center}
			\begin{Huge}
				完$\ $结$\ $撒$\ $花\\
			\end{Huge}
		\end{center}
	\end{frame}

\end{document}